\documentclass{report}
% подключаем русский шрифт
\usepackage[utf8]{inputenc}
\usepackage[russian]{babel}
\righthyphenmin=2
\voffset = 0pt
\topmargin = 0pt
\headsep = 0pt
\headheight = 0pt
\textheight = 800pt
\oddsidemargin = 31pt
\marginparsep = 0pt
\marginparwidth = 0pt
\textwidth = 500pt

% начинаем документ
\begin{document}
\section* {\bfseries Maze\_CPP}
\section* {\bfseries Краткое описание возможностей программы.}
\begin{itemize}
\item[1)] В программе предусмотрена возможность загрузки лабиринта из файла и сохранение в файл

\item[2)] Предусмотрены настройки цвета фона, стен лабиринта

\item[3)] Максимальный размер лабиринта - 50х50

\item[4)] Загруженный лабиринт отрисовывается на экране в поле размером 500 x 500

\item[5)] Толщина «стены» - 2 пикселя

\item[6)] Возможность автоматической генерации идеального лабиринта по заданным размерам

\item[7)] Возможность показать решение любого лабиринта, который изображен на экране программы с помощью задания начальной и конечной точки
\end{itemize}
\section* {\bfseries Установка.}

Чтобы начать использовать программу Maze, его необходимо установить с помощью команды make install. Эта команда создает папку Maze\_CPP в текущей директории и устанавливает программму в неё.

\section* {\bfseries Удаление.}

Удалить приложение можно с помощью команды make uninstall.

\section* {\bfseries Архивация.}

Архивировать прорамму можно с помощью команды make dist. Формат архива - tar.gz

\section* {\bfseries Основные особенности:}
\begin{itemize}
\item Загрузка и просмотр лабиринта из файла:

Чтобы открыть файл нажмите File-> Open file...

Лабиринт может храниться в файле в виде количества строк и столбцов, а также двух матриц, содержащих положение вертикальных и горизонтальных стен соответственно.

В первой матрице отображается наличие стены справа от каждой ячейки, а во второй - снизу.

Пример подобного файла:


4 4

0 0 0 1

1 0 1 1

0 1 0 1

0 0 0 1



1 0 1 0

0 0 1 0

1 1 0 1

1 1 1 1


\item  Лабиринт генерируется  согласно алгоритму Эллера;

\item Пользователем вводится только размерность лабиринта: количество строк и столбцов;

\item Для отображения кратчайшего пути решения лабиринта кликните на две точки (начальная и конечная) на отображенном лабиринте;

\item Сгенерированный лабиринт может сохраняться в файл в формате, описанном выше по кнопке File-> Save file...;
\end{itemize}
\end{document}